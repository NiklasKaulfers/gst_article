\documentclass{report}
\usepackage[utf8]{inputenc}
\usepackage[T1]{fontenc}
\usepackage[ngerman]{babel}
\usepackage{graphicx,wrapfig,lipsum}
\usepackage{geometry}
\usepackage{minted}
\usepackage{xcolor}
\usepackage{csquotes}
\usepackage{dirtree}
\usepackage{hyperref}
\usepackage{fancyhdr}
\usepackage[toc,page]{appendix}
\usepackage[backend=biber,style=ieee]{biblatex}  % or style=numeric, authoryear, etc.
% Bibliography file: updated to use `main.bib` (contains your entries)
\addbibresource{main.bib}
\geometry{top=2.5cm, bottom=2.5cm, left=2.5cm, right=2.5cm}



% -------------------------------------------------------
% Titelseiten-Informationen
% -------------------------------------------------------
\title{Softwaretests in dem View-Gewinnt-Projekt der iib23}

% Art der Arbeit + Studiengang
\newcommand{\arbeitstyp}{Beleg}
% Modul / Lehrveranstaltung + Dozent:innen
\newcommand{\modul}{Modul / Lehrveranstaltung: Grundlagen des Softwaretestens\\
Dozent: Prof.\ Dr.\ Matthias Längrich}
% Hochschule + Fakultät
\newcommand{\hochschule}{
\includegraphics[height=2.5cm]{LOGO_HSZG_SUBLINE_GRANIT.png}\\[5mm]   % Logo der HSZG einfügen
Hochschule Zittau/Görlitz \\ Fakultät Elektrotechnik und Informatik
}

% Autor
\author{
    Niklas Kaulfers \\[2mm]
    Matrikelnummer: 1064032
}
% Betreuer (für BA-Thesis, sonst leer lassen)
\newcommand{\betreuer}{}

% Datum der Abgabe
\date{Abgabedatum: \today}

% -------------------------------------------------------
% Titelseite neu definieren
% -------------------------------------------------------
\makeatletter
\renewcommand{\maketitle}{
    \begin{titlepage}
        \centering
        {\Large \hochschule\par}
        \vfill

        {\huge \bfseries \@title\par}
        \vspace{1cm}

        {\large \arbeitstyp\par}
        \vspace{0.5cm}

        {\large \modul\par}
        \vspace{0.5cm}

        {\large \betreuer\par}
        \vspace{1.5cm}

        {\Large \@author\par}
        \vspace{2cm}

        {\large \@date\par}

        \vfill
    \end{titlepage}
}
\makeatother


\fancypagestyle{plain}{%
    \fancyhead[C]{%
        \includegraphics[height=1.5cm]{LOGO_HSZG_SUBLINE_GRANIT_ICON_ONLY.png}%
    }
}

\begin{document}
\pagestyle{fancy}

\setlength{\headheight}{47pt}

\maketitle

\tableofcontents


\chapter{Einleitung}
{
    Fehler sind in der Softwareentwicklung nahezu nicht zu vermeiden, 
    jedoch können Entwickler versuchen diesen so gut wie möglich vorzubeugen und somit nicht in entsprechende Produktionsumgebungen vordringen zu lassen. 
    Dies tun sie mittels Softwaretests. In dieser Arbeit wird sich mit der Durchführung dieser Tests an einem Projekt auseinandergesetzt.
\par}

\chapter{Aufbau}
\section{Testobjekt}
    Das Projekt, an welchem die Tests durchgeführt wurden ist ein Gemeinschaftsprojekt des Matrikels iib23 aus dem Modul OOP (Objektorientierte Programmierung). 
    Es handelt sich hierbei um eine Implementation des Spiels Vier Gewinnt, einem bekannten und relativ leicht verständlichem Strategiespiel. 
    Dieses Spiel zu zweit Spielen spielbar. 
    Das Ziel ist es jeweils 4 der eigenen Steine in einer Reihe zu platzieren und mit eigenen Steinen verhindern, dass der Gegner dasselbe tut.\\
    Die Studenten des Moduls wurden im Rahmen des Moduls in verschiedene Teams aufgeteilt um das Projekt gemeinsam anzufertigen.
    Die Teams waren Projektmanagment, Backend, Frontend und Testing (Ich war Teil des Projektmanagmentteams). 
    Im Fall des Projektes gibt es Möglichkeiten, zum lokalen spielen gegen einen anderen Spieler oder gegen einen Computergegner. 
    Das Spiel hat zudem ein GUI (Graphical User Interface), 
    welches den Spielstand visualisiert und entsprechende Aktionen für den Spieler erlaubt\footnote{\ref{img:gameInProgress}, Screenshot aus dem Spiel}.\\
    Zur Erstellung wurde über \href{https://github.com/NiklasKaulfers/VierGewinnt}{\underline{GitHub}} kollaboriert.
    Alle im Rahmen dieser Arbeit erbrachten Leistungen sind in dieser GitHub-Repository zu finden.
\subsection{Psychologische Betrachtung}
    Tests werden häufig als unnötig und als Methode zum zeigen, dass es keine Errors gibt angesehen\footfullcite[p.10]{myers2004art}.
    Tatsächlich werden Tests jedoch verwendet um Errors zu finden und fehlverhalten der Software im Vorhinein zu verringern.
    Auch hier im Projekt waren die bereits existieren Tests eher minimal und testen vor allem die Hauptszenarios. 
    Vor allem explizite Logik ist noch ungetestet.
\subsection{Ökonomische Betrachtung}
    Während im ökonomischen Sinn meist die Gesamtkosten betrachtet werden, ist in dieser Arbeit die Zeit von größerer Bedeutung.
    Rahmen des Projektes stellt das Wintersemester 2025/26.\\
    Die Tests müssen deswegen in einem sehr gerigem Zeitrahmen gerfertigt werden.\\
    Das Projekt war ein Teil eines Hochschulmoduls, somit waren die Ressourcen stark begrenzt. 
    Auch hierzu anzumerken ist, dass es sich um ein Modul aus dem zweiten Semester handelt, 
    weswegen auch die Erfahrungen der beteiligten noch nicht ausgereift waren.





\section{Konzept der Tests}
    Trotz der verhältnismäßig kleinen Größe des Projektes eignet es sich sehr gut zum Testen. 
    Die Vorgehensweise ist wie folgt.
    Zuerst werden Unit-Tests von den wichtigesten Teilen der Software durchgeführt. 
    So beispielsweise die Methode des platzieren eines Steines auf dem Spielfeld.
    Auch andere Methoden wie die Erstellung von Speicherdaten kann überprüft werden. 
    Hierfür wird im Projekt eine eigene Methode verwendet.

\section{Nötige Anpassungen}
    Das Projekt auszuführen stellte sich als unnötig komplex heraus, 
    da das Mainfile, welches die Software startet, in einem von vielen Unterordnern aufzufinden war.\\
    Im vorhinein war das Projekt sehr unübersichtlich und benötigte einige Verbesserungen, vor allem in Hinblick auf Namensgebung von Variablen.
    Diese waren Teils mit einzelnen Buchstaben benannt und somit schwer verständlich und lesbar\footnote{\ref{gph:startProjectLayout}, Layout des Projektes zu Beginn der Tests}.\\
    Hier ist eine bessere Projektstruktur, welche eine einfachere Ausführung und Verarbeitung ermöglicht benötigt.
    Um dies zu verwirklichen wurde das Projekt in einer neuen Branch in \href{https://github.com/NiklasKaulfers/VierGewinnt/tree/gradle-test}{\underline{gradle}} neu aufgesetzt.
    Nun kann das Projekt einfach über %\mintinline{shell}|
    gradle run
    %|
    ausgeführt werden.
    Auch sämtliche Tests sind mit gradle test somit leichter zugänglich.

\chapter{Durchführung}
\section{Unit Tests}
\section{Integration Tests}
\section{System Testing}
\section{Acceptance Testing}

\chapter{Probleme des Projektes}
\section{Behobene Probleme}
\section{Nicht behobene Probleme}

\chapter{Fazit}

\begin{appendices}
\chapter{Grafiken}

    \begin{wrapfigure}{r}{\linewidth}
        \caption{Bild des Spiels}\label{img:gameInProgress}
        \includegraphics[width=\linewidth]{screenshot_game_inprogress.png}
    \end{wrapfigure}

    \begin{figure}
        \caption{Projektstruktur vor Verbesserungen (Einstiegspunkt in rot)}\label{gph:startProjectLayout}
        \dirtree{%
        .1 VierGewinnt/.
        .2 lib/.
        .3 junit-platform-console-standalone.jar.
        .2 res/.
        .3 SpielerVsComputer.png.
        .3 SpielerVsSpieler.png.
        .2 src/.
        .3 api/.
        .4 BoardInterface.java.
        .4 BoardTestInterface.java.
        .4 TileInterface.java.
        .3 gui/.
        .4 entity/.
        .5 BordersForCircle.java.
        .5 Circle.java.
        .4 frames/.
        .5 EndFrame.java.
        .5 MainPanel.java.
        .5 OptionsFrame.java.
        .5 StartFrame.java.
        .4 handler/.
        .5 MouseHandler.java.
        .4 main/.
        .5 \textcolor{red}{App.java}.
        .3 logic/.
        .4 Board.java.
        .4 Tile.java.
        .3 test/.
        .4 GameTest.java.
        .2 {.gitignore}.
        .2 README.md.
        .2 TEST.md.
        .2 VierGewinnt.jar.
        .2 win4.jar.
        }
    \end{figure}


\end{appendices}

\printbibliography


\end{document}
