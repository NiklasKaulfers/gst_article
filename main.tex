\documentclass{report}
\usepackage{float}
\usepackage[utf8]{inputenc}
\usepackage[T1]{fontenc}
\usepackage[ngerman]{babel}
\usepackage{graphicx,wrapfig,lipsum}
\usepackage{geometry}
\usepackage{minted}
\usepackage{xcolor}
\usepackage{tikz}
\usepackage{csquotes}
\usepackage{dirtree}
\usepackage{hyperref}
\usepackage{fancyhdr}
\usepackage{makecell}
\usepackage{array}
\usepackage[toc,page]{appendix}
\usepackage[backend=biber,style=ieee]{biblatex}  % or style=numeric, authoryear, etc.
\addbibresource{main.bib}
\geometry{top=2.5cm, bottom=2.5cm, left=2.5cm, right=2.5cm}



% -------------------------------------------------------
% Titelseiten-Informationen
% -------------------------------------------------------
\title{Softwaretests in dem Vier-Gewinnt-Projekt der IIb23}

% Art der Arbeit + Studiengang
\newcommand{\arbeitstyp}{Beleg}
% Modul / Lehrveranstaltung + Dozent:innen
\newcommand{\modul}{Modul / Lehrveranstaltung: Grundlagen des Softwaretestens\\
Dozent: Prof.\ Dr.\ Matthias Längrich}
% Hochschule + Fakultät
\newcommand{\hochschule}{
\includegraphics[height=2.5cm]{LOGO_HSZG_SUBLINE_GRANIT.png}\\[5mm]
Hochschule Zittau/Görlitz \\ Fakultät Elektrotechnik und Informatik
}

% Autor
\author{
    Niklas Kaulfers \\[2mm]
    Matrikelnummer: 1064032
}
% Betreuer (für BA-Thesis, sonst leer lassen)
\newcommand{\betreuer}{}

% Datum der Abgabe
\date{Abgabedatum: \today}

% -------------------------------------------------------
% Titelseite neu definieren
% -------------------------------------------------------
\makeatletter
\renewcommand{\maketitle}{
    \begin{titlepage}
        \centering
        {\Large \hochschule\par}
        \vfill

        {\huge \bfseries \@title\par}
        \vspace{1cm}

        {\large \arbeitstyp\par}
        \vspace{0.5cm}

        {\large \modul\par}
        \vspace{0.5cm}

        {\large \betreuer\par}
        \vspace{1.5cm}

        {\Large \@author\par}
        \vspace{2cm}

        {\large \@date\par}

        \vfill
    \end{titlepage}
}
\makeatother


\fancypagestyle{plain}{%
    \fancyhead[C]{%
        \includegraphics[height=1.25cm]{LOGO_HSZG_SUBLINE_GRANIT_ICON_ONLY.png}%
    }
    \renewcommand{\footrulewidth}{0.4pt}
}

\begin{document}
\pagestyle{plain}
\pagenumbering{gobble}

\setlength{\headheight}{47pt}

\maketitle

\tableofcontents

\pagenumbering{arabic}

\fancyfoot[L]{\sffamily Kaulfers}
\fancyfoot[C]{\sffamily Belegarbeit}
\fancyfoot[R]{\sffamily\thepage}

\chapter{Einleitung}
{
    Fehler sind in der Softwareentwicklung nahezu nicht zu vermeiden, 
    jedoch können Entwickler versuchen diesen so gut wie möglich vorzubeugen und somit nicht in entsprechende Produktionsumgebungen vordringen zu lassen. 
    Dies tun sie mittels Softwaretests. In dieser Arbeit wird sich mit der Durchführung dieser Tests an einem Projekt auseinandergesetzt.\\
    Im Verlauf der Arbeit sind \underline{unterstrichene} und \textit{kursive} Textabschnitte zu finden. 
    \underline{Unterstrichene} sind links und \textit{kursive} shell anweisungen. 
\par}

\chapter{Testobjekt}
\section{Vorstellung des Projektes}
    Das Projekt, an welchem die Tests durchgeführt wurden ist ein Gemeinschaftsprojekt des Matrikels IIb23 aus dem Modul OOP (Objektorientierte Programmierung). 
    Es handelt sich hierbei um eine Java-Implementation des Spiels Vier Gewinnt, einem bekannten und relativ leicht verständlichem Strategiespiel. 
    Dieses Spiel zu zweit Spielen spielbar. 
    Das Ziel ist es jeweils 4 der eigenen Steine in einer Reihe zu platzieren und mit eigenen Steinen verhindern, dass der Gegner dasselbe tut.\\
    Die Studenten des Moduls wurden im Rahmen des Moduls in verschiedene Teams aufgeteilt um das Projekt gemeinsam anzufertigen.
    Die Teams waren Projektmanagment, Backend, Frontend und Testing (Ich war Teil des Projektmanagmentteams).
    Zur Erstellung wurde über \href{https://github.com/NiklasKaulfers/VierGewinnt}{\underline{GitHub}} kollaboriert.
    Alle im Rahmen dieser Arbeit erbrachten Leistungen sind in dieser GitHub-Repository zu finden.
    Im Fall des Projektes gibt es Möglichkeiten, zum lokalen spielen gegen einen anderen Spieler oder gegen einen Computergegner. 
    Das Spiel hat zudem ein GUI (Graphical User Interface), 
    welches den Spielstand visualisiert und entsprechende Aktionen für den Spieler erlaubt\footnote{\ref{img:gameInProgress}, Screenshot aus dem Spiel}.\\
    Im Hintergrund besteht das Spiel aus 4 Komponenten; Api, Logik, Frontend und Tests.
    Diese sind voneinander logisch in der Ordnerstruktur abgetrennt\footnote{\ref{gph:startProjectLayout}, Layout des Projektes zu Beginn der Tests}.
    Der api Ordern beinhaltet Interfaces, welche die Logik definieren. 
    Hier gibt es auch ein Interface, welches ausschließlich zum Testen gemacht wurde. 
    Die Logik implementiert diese Interfaces und versorgt sie mit Funkionalität.
    Für ein ordentliches grafisches Display sorgt das Frontend. 
    Dieses ist im Vergleich zur Logik deutlich bessser Modualisiert und somit auch besser lesbar.
    Schlussendlich gibt es in dem Projekt bereits Tests, welche im Test-Ordner sind.
    Jedoch waren diese vor dieser Arbeit noch nicht ausgereift.
 
\subsection{Psychologische Betrachtung}
    Tests werden häufig als unnötig und als Methode zum zeigen, dass es keine Errors gibt angesehen.
    Tatsächlich werden Tests jedoch verwendet um Errors zu finden und fehlverhalten der Software im Vorhinein zu verringern\footfullcite[p.10]{myers2004art}.
    Auch hier im Projekt waren die bereits existieren Tests eher minimal und testen vor allem die Hauptszenarios. 
    Vor allem explizite Logik ist noch ungetestet.\\
    Implementation von Verbesserungen oder eventuelle Bugfixes sind zudem unerwartet, da es sich hierbei um ein abgeschlossenes Projekt handelt.
    Das Projekt hat keinerlei Bugreportmethoden oder ähnliches. Es gibt zwar Issues in der GitHub-Repository, diese werden jedoch nicht in Pullrequests thematisiert.
\subsection{Ökonomische Betrachtung}
    Während im ökonomischen Sinn meist die Gesamtkosten betrachtet werden, ist in dieser Arbeit die Zeit von größerer Bedeutung.
    Rahmen des Projektes stellt das Wintersemester 2025/26.\\
    Die Tests müssen deswegen in einem sehr gerigem Zeitrahmen gerfertigt werden.\\
    Das Projekt war ein Teil eines Hochschulmoduls, somit waren die Ressourcen stark begrenzt. 
    Auch hierzu anzumerken ist, dass es sich um ein Modul aus dem zweiten Semester handelt, 
    weswegen auch die Erfahrungen der beteiligten noch nicht ausgereift waren.
\section{Nötige Anpassungen}
    Das Projekt auszuführen stellte sich als unnötig komplex heraus, 
    da das Mainfile, welches die Software startet, in einem von vielen Unterordnern aufzufinden war.\\
    Im vorhinein war das Projekt sehr unübersichtlich und benötigte einige Verbesserungen, vor allem in Hinblick auf Namensgebung von Variablen.
    Diese waren Teils mit einzelnen Buchstaben benannt und somit schwer verständlich und lesbar\footnote{\ref{gph:startProjectLayout}, Layout des Projektes zu Beginn der Tests}.\\
    Hier ist eine bessere Projektstruktur, welche eine einfachere Ausführung und Verarbeitung ermöglicht benötigt.
    Um dies zu verwirklichen wurde das Projekt in einer neuen Branch in \href{https://github.com/NiklasKaulfers/VierGewinnt/tree/gradle-test}{\underline{gradle}} neu aufgesetzt.
    Nun kann das Projekt einfach über \textit{gradle run} ausgeführt werden.
    Auch sämtliche Tests sind mit \textit{gradle test} somit leichter zugänglich.
\section{Eignung zum Testen}
    Trotz der verhältnismäßig kleinen Größe des Projektes eignet es sich sehr gut zum Testen. 
    Die Vorgehensweise ist wie folgt.
    Zuerst werden Unit-Tests von den wichtigesten Teilen der Software durchgeführt. 
    So beispielsweise die Methode des platzieren eines Steines auf dem Spielfeld.
    Auch andere Methoden wie die Erstellung von Speicherdaten kann überprüft werden. 
    Hierfür wird im Projekt eine eigene Methode verwendet.


\chapter{Planung}
\section{Arten von Tests}
\section{Qualitätsplanung}
    Um eine hohe Qualität der Tests sicherzustellen sollten gewisse Elemente und die damit verbundenen Anforderungen vor der tatsächlichen Umsetzung der Tests aufgelistet werden\footfullcite[pp.93]{baresi2006introduction}.
    Eine solche Qualitätsplanung ist auch hier von Sinn\footnote{\ref{gph:testQualityPlan}, Qualitätsplanung für Board}.
    Diese Planung ist nur für Board vorgenommen wurden um in einem sinngemäßen Rahmen zu bleiben.  

\chapter{Durchführung}
    Zur Durchführung wird jeweils individuell ein Testfall detailiert erläutert.
\section{Unit Tests - verbleibende Plätze in einer Spalte}
    Um zu schauen, ob weiteres platzieren von Steinen in einer Spalte erlaubt ist gibt es eine Funktion \texttt{int isTopOfColumn(int column)}.
    Diese überprüft, wieviele Plätze noch in einer Spalte frei sind.\\
    Wenn man diese Funktion betrachtet fällt zuerst der Name auf. 
    Mit \texttt{isTopOfColumn} ist ein boolean-Wert als Ausgabe zu erwarten.
    Jedoch ist der tatsächliche Rückgabewert vom Typen int.\\
    Zur Implementierung der Tests müssen zuerst Grenzwerte gesetzt werden, in welchen sich die Funktion wie verhalten soll.
    Das Standartboard hat ein Layout von 7x7 somit kommt Abbildung \ref{gph:validInputIsTopOfColumn} für die validen inputs der Funktion zustande.
    Daraus folgt, dass Tests für die Inputs -1, 0, 6, 7 durchgeführt werden müssen.
    Die Funktion ist nicht nur von der tatsächlichen Spalte abhängig, sondern auch von der Anzahl der platzierten Steine.
    Somit wird auch getestet, ob die richtigen Werte innerhalb des gültigen Bereichs zurückgegeben werden.
    Um hier das genaue Verhalten zu testen fehlt im Code eine Möglichkeit, um Steine über oder unter dem Spielfeld zu plazieren.
    \begin{figure}[h!]
    \centering
        \begin{tikzpicture}[>=stealth, thick]

            \draw[->] (-3,0) -- (9.5,0) node[right]{Spalte};

            \draw[ultra thick] (0,0.2) -- (0,-0.2) node[above] at (0,0.1) {Untergrenze};
            \draw[ultra thick] (6,0.2) -- (6,-0.2) node[above] at (6,0.1) {Obergrenze};
            
            \foreach \x in {-3,-2,-1,0,1,2,3,4,5,6,7,8,9} {
                \draw (\x,0.15) -- (\x,-0.15) node[below]{\x};
            }

            \draw[very thick, green!60!black] (0,0) -- (6,0);

            \node[above] at (3,0.1) {Gültiger Bereich};

        \end{tikzpicture}
        \caption{Grenzwerte der Inputs von \texttt{isTopOfColumn()} (im Fall des Boardlayouts 7x7)}
        \label{gph:validInputIsTopOfColumn}
    \end{figure}\\
    In der Durchführung der Tests stellt sich heraus, dass das Verhalten unterhalb der Untergrenze und überhalb der Obergrenze nicht definiert ist.
    Somit kommt es zu einer \texttt{ArrayIndexOutOfBoundsException} für beide Fälle.
    Da es keine Möglichkeit gibt, die Steine so zu platzieren, dass in einer Spalte mehr Steine liegen als vorgesehen, ist es auch nicht möglich das Verhalten in diesem Fall zu testen.
    Für alle anderen Fälle verhält sich die Funktion wie vorgesehen\footnote{vgl.: https://github.com/NiklasKaulfers/VierGewinnt/blob/main/src/test/logic/board/IsTopOfColumnTest.java}.

    
\section{Integration Tests - Erstellung und Abspeicherung eines Speichercodes}
    Das Spiel hat eine Speicherlogik. 
    Für diese wird ein String erstellt, welcher den aktuellen Stand des Spieles aufzeichnet.
    Dieser funktioniert wie folgt:\verb!{playerTurn}a{rows}a{columns}a{isFull}aB{boardLayout}!.
    Dieser String wird dann in einem txt-File gespeichert und kann beim neustarten des Spiels aufgerufen werden.
    Für den Test ist es wichtig, dass sich das txt-File ordentlich öffnet und ausliest.
    Zudem muss getestet werden, ob der Speichercode den Spielstand ordnungsgemäß darstellt.\\
\section{System Testing}
\section{Acceptance Testing}

\chapter{Probleme des Projektes}
\section{Behobene Probleme}
\section{Nicht behobene Probleme}

\chapter{Fazit}

\begin{appendices}
\chapter{Grafiken}
    \begin{figure}
        \caption{Bild des Spiels}\label{img:gameInProgress}
        \includegraphics[width=\linewidth]{screenshot_game_inprogress.png}
    \end{figure}
    \newpage
    \begin{figure}
        \caption{Projektstruktur vor Verbesserungen (Einstiegspunkt in rot)}\label{gph:startProjectLayout}
        \dirtree{%
        .1 VierGewinnt/.
        .2 lib/.
        .3 junit-platform-console-standalone.jar.
        .2 res/.
        .3 SpielerVsComputer.png.
        .3 SpielerVsSpieler.png.
        .2 src/.
        .3 api/.
        .4 BoardInterface.java.
        .4 BoardTestInterface.java.
        .4 TileInterface.java.
        .3 gui/.
        .4 entity/.
        .5 BordersForCircle.java.
        .5 Circle.java.
        .4 frames/.
        .5 EndFrame.java.
        .5 MainPanel.java.
        .5 OptionsFrame.java.
        .5 StartFrame.java.
        .4 handler/.
        .5 MouseHandler.java.
        .4 main/.
        .5 \textcolor{red}{App.java}.
        .3 logic/.
        .4 Board.java.
        .4 Tile.java.
        .3 test/.
        .4 GameTest.java.
        .2 {.gitignore}.
        .2 README.md.
        .2 TEST.md.
        .2 VierGewinnt.jar.
        .2 win4.jar.
        }
    \end{figure}
    \newpage
\chapter{Tabellen}
    \begin{figure}
        \caption{Planung der Tests für Board.java (nach \fullcite[pp.93]{baresi2006introduction})}\label{gph:testQualityPlan}
        \begin{center}
            \begin{tabular}{ | c | c | }
                \hline
                Test Items & Board.java \\
                \hline
                Features to be tested & placeStone, computer algorithm, save/load, winning \\
                \hline
                Features not to be tested & Setters/Getters \\
                \hline
                Approach & \makecell{Unit Tests mit Grenzwerten oder\\
                    Äquivalenzklassen für zu umfangreichen Code\\
                    Integration Tests für einen Spielablauf mit Use Case\\
                    KI-basiertes Testen für höhere Effizienz bei redundanten Fällen}\\
                \hline
                Pass/Fail cirteria & \makecell{Alle Funktionen verhalten sich wie erwartet\\
                    Fail bei unter 100\% Success}\\
                \hline 
                Suspension and resumption criteria & 
                    \makecell{Codequalität zu niedrig bzw unlesbar:\\
                    Codeverbesserungen sind umgesetzt\\ \\
                    Kritischer Error/Crash:\\
                    Fix des kritischen Errors}
                    \\
                \hline
                Risks and contingencies & Zeitliche Begrenzungen \\
                \hline
                Deliverables & Bugs und generelle QA \\
                \hline 
                Tasks \& Schedule & Zeit bis Februar 2025 \\
                \hline 
                Staff \& responsibilities & \makecell{Niklas Kaulfers (Umsetzung),\\IIb23 (Bereitstellung des Projekts und Softwareentwicklung)} \\
                \hline 
                Environment needs & JUnit, Java correto 23, IntelliJ IDEA \\
                \hline
            \end{tabular}
        \end{center}
    \end{figure}
    \newpage
\end{appendices}
\printbibliography\end{document}