\documentclass{report}
\usepackage[utf8]{inputenc}
\usepackage[T1]{fontenc}
\usepackage[ngerman]{babel}
\usepackage{graphicx,wrapfig,lipsum}
\usepackage{geometry}
\usepackage{minted}
\usepackage{xcolor}
\usepackage{csquotes}
\usepackage{dirtree}
\usepackage{hyperref}
\usepackage[toc,page]{appendix}
\usepackage[backend=biber,style=ieee]{biblatex}  % or style=numeric, authoryear, etc.
\addbibresource{references.bib}
\geometry{top=2.5cm, bottom=2.5cm, left=2.5cm, right=2.5cm}


% -------------------------------------------------------
% Titelseiten-Informationen
% -------------------------------------------------------
\title{Softwaretests in dem View-Gewinnt-Projekt der iib23}

% Art der Arbeit + Studiengang
\newcommand{\arbeitstyp}{Beleg}
% Modul / Lehrveranstaltung + Dozent:innen
\newcommand{\modul}{Modul / Lehrveranstaltung: Grundlagen des Softwaretestens\\
Dozent: Prof.\ Dr.\ Matthias Längrich}
% Hochschule + Fakultät
\newcommand{\hochschule}{
\includegraphics[height=2.5cm]{LOGO_HSZG_SUBLINE_GRANIT.png}\\[5mm]   % Logo der HSZG einfügen
Hochschule Zittau/Görlitz \\ Fakultät Elektrotechnik und Informatik
}

% Autor
\author{
    Niklas Kaulfers \\[2mm]
    Matrikelnummer: 1064032
}
% Betreuer (für BA-Thesis, sonst leer lassen)
\newcommand{\betreuer}{}

% Datum der Abgabe
\date{Abgabedatum: \today}

% -------------------------------------------------------
% Titelseite neu definieren
% -------------------------------------------------------
\makeatletter
\renewcommand{\maketitle}{
    \begin{titlepage}
        \centering
        {\Large \hochschule\par}
        \vfill

        {\huge \bfseries \@title\par}
        \vspace{1cm}

        {\large \arbeitstyp\par}
        \vspace{0.5cm}

        {\large \modul\par}
        \vspace{0.5cm}

        {\large \betreuer\par}
        \vspace{1.5cm}

        {\Large \@author\par}
        \vspace{2cm}

        {\large \@date\par}

        \vfill
    \end{titlepage}
}
\makeatother

\begin{document}

\maketitle

\tableofcontents


\chapter{Einleitung}
{
Fehler sind in der Softwareentwicklung nahezu nicht zu vermeiden, 
jedoch können Entwickler versuchen diesen so gut wie möglich vorzubeugen und somit nicht in entsprechende Produktionsumgebungen vordringen zu lassen. 
Dies tun sie mittels Softwaretests. In dieser Arbeit wird sich mit der Durchführung dieser Tests an einem Projekt auseinandergesetzt.
\par}

\chapter{Aufbau}
\section{Testobjekt}
Das Projekt, an welchem die Tests durchgeführt wurden ist ein Gemeinschaftsprojekt des Matrikels iib23 aus dem Modul OOP (Objektorientierte Programmierung). 
Es handelt sich hierbei um eine Implementation des Spiels Vier Gewinnt, einem bekannten und relativ leicht verständlichem Strategiespiel. Dieses Spiel zu zweit Spielen spielbar. 
Im Fall des Projektes gibt es Möglichkeiten, zum lokalen spielen gegen einen anderen Spieler oder gegen einen Computergegner. 
Das Spiel hat zudem ein GUI (Graphical User Interface), 
welches den Spielstand visualisiert und entsprechende Aktionen für den Spieler erlaubt\ref{img:gameInProgress}.\\
Zur Erstellung wurde über \href{https://github.com/NiklasKaulfers/VierGewinnt}{\underline{GitHub}} 
kollaboriert. Da es sich hier um ein Projekt von Studenten handelt, sind mehrere Fehler vorhanden.




\section{Konzept der Tests}
Trotz der verhältnismäßig kleinen Größe des Projektes eignet es sich sehr gut zum Testen. Die Vorgehensweise ist wie folgt.
Zuerst werden Unit-Tests von den wichtigesten Teilen der Software durchgeführt. So beispielsweise die Methode des platzieren eines Steines auf dem Spielfeld.
Auch andere Methoden wie die Erstellung von Speicherdaten kann überprüft werden. Hierfür wird im Projekt eine eigene Methode verwendet.
\section{Nötige Anpassungen}
Das Projekt auszuführen stellte sich als unnötig komplex heraus, 
da das Mainfile, welches die Software startet, in einem von vielen Unterordnern aufzufinden war.\\
Im vorhinein war das Projekt sehr unübersichtlich und benötigte einige Verbesserungen, vor allem in Hinblick auf Namensgebung von Variablen.
Diese waren Teils mit einzelnen Buchstaben benannt und somit schwer verständlich und lesbar\ref{gph:startProjectLayout}.

\chapter{Durchführung}
\section{Unit Tests}
\section{Integration Tests}
\section{UI Tests}

\chapter{Probleme des Projektes}
\section{Behobene Probleme}
\section{Nicht behobene Probleme}

\chapter{Fazit}

\begin{appendices}
\chapter{Grafiken}

\begin{wrapfigure}{r}{\linewidth}
\includegraphics[width=\linewidth]{screenshot_game_inprogress.png}
\caption{Bild des Spiels}\label{img:gameInProgress}


\end{wrapfigure}
\begin{figure}
\dirtree{%
.1 VierGewinnt/.
.2 lib/.
.3 junit-platform-console-standalone.
.2 res/.
.3 SpielerVsComputer.png.
.3 SpielerVsSpieler.png.
.2 src/.
.3 api/.
.4 BoardInterface.
.4 BoardTestInterface.
.4 TileInterface.
.3 gui/.
.4 entity/.
.5 BordersForCircle.
.5 Circle.
.4 frames/.
.5 EndFrame.
.5 MainPanel.
.5 OptionsFrame.
.5 StartFrame.
.4 handler/.
.5 MouseHandler.
.4 main/.
.5 \textcolor{red}{App}.
.3 logic/.
.4 Board.
.4 Tile.
.3 test/.
.4 GameTest.
.2 {.gitignore}.
.2 README.md.
.2 TEST.md.
.2 VierGewinnt.jar.
.2 win4.jar.
}
\caption{Projektstruktur vor Verbesserungen (Einstiegspunkt in rot)}\label{gph:startProjectLayout}
\end{figure}


\end{appendices}



\end{document}
